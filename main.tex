\documentclass[12pt,a4paper]{article}
\usepackage[polish]{babel}
\usepackage[T1]{fontenc}
\usepackage[]{algorithm2e}
\usepackage{listings}
\usepackage{amsmath}
\usepackage{graphicx}

\usepackage{color}
\usepackage{listings}

\lstloadlanguages{% Check Dokumentation for further languages ...
	HTML,
	Java,
        python
}

\lstdefinelanguage{batch}{
  morekeywords={@echo, echo, pause, menu, set, if, call run_project, else if, else, show_info, exit, goto, off, run_project, setlocal, enabledelayedexpansion, run_project_menu, type, backing_up, endlocal, start},
  sensitive=false,
  morecomment=[l]{REM},
}

\lstdefinestyle{batch}{
  language=batch,
  basicstyle=\small\ttfamily,
  commentstyle=\color{green!40!black},
  keywordstyle=\color{blue},
  numberstyle=\tiny\color{black},
  numbers=left,
  stepnumber=1,
  numbersep=5pt,
  backgroundcolor=\color{white},
  frame=none,
  rulecolor=\color{black},
  tabsize=2,
  captionpos=b,
  breaklines=true,
  showspaces=false,
  showstringspaces=false,
}

\lstdefinelanguage{JavaScript}{
  keywords={typeof, new, true, false, catch, function, return, null, catch, switch, var, if, in, while, do, else, case, break},
  keywordstyle=\color{blue}\bfseries,
  ndkeywords={class, export, boolean, throw, implements, import, this},
  ndkeywordstyle=\color{black}\bfseries,
  identifierstyle=\color{red},
  sensitive=false,
  comment=[l]{//},
  morecomment=[s]{/*}{*/},
  commentstyle=\color{purple}\ttfamily,
  stringstyle=\color{red}\ttfamily,
  morestring=[b]',
  morestring=[b]"
}

\lstdefinestyle{JavaScript}{
  language=JavaScript,
  basicstyle=\small\ttfamily,
  numbers=left,
  numberstyle=\tiny,
  numbersep=5pt,
  backgroundcolor=\color{white},
  showspaces=false,
  showstringspaces=false,
  tabsize=2,
  breaklines=true,
  captionpos=b,
  frame=none,
}

\lstdefinelanguage{CSS}{
  keywords={color, background, font-size, margin, padding, border, display, width, height, position, top, left, right, bottom, text, align, body, font, family, h1, table, collapse, margin-top, th, td, tr, hover, datetime},
  sensitive=true,
  morecomment=[s]{/*}{*/},
  morecomment=[l]://,
  morestring=[b]',
  morestring=[b]",
}

\lstdefinestyle{CSS}{
  language=CSS,
  basicstyle=\small\ttfamily,
  numbers=left,
  numberstyle=\tiny,
  numbersep=5pt,
  backgroundcolor=\color{white},
  showspaces=false,
  showstringspaces=false,
  tabsize=2,
  breaklines=true,
  captionpos=b,
  frame=none,
  keywordstyle=\color{blue}\bfseries,
  commentstyle=\color{green!40!black}\itshape,
  stringstyle=\color{red}\ttfamily,
}


\definecolor{red}{rgb}{0.6,0,0} % for strings
\definecolor{blue}{rgb}{0,0,0.6}
\definecolor{green}{rgb}{0,0.8,0}
\definecolor{cyan}{rgb}{0.0,0.6,0.6}

\lstset{
	language=csh,
	basicstyle=\footnotesize\ttfamily,
	numbers=left,
	numberstyle=\tiny,
	numbersep=5pt,
	tabsize=2,
	extendedchars=true,
	breaklines=true,
	frame=b,
	stringstyle=\color{blue}\ttfamily,
	showspaces=false,
	showtabs=false,
	xleftmargin=17pt,
	framexleftmargin=17pt,
	framexrightmargin=5pt,
	framexbottommargin=4pt,
	commentstyle=\color{green},
	morecomment=[l]{//}, %use comment-line-style!
	morecomment=[s]{/*}{*/}, %for multiline comments
	showstringspaces=false,
	morekeywords={ abstract, event, new, struct,
		as, explicit, null, switch,
		base, extern, object, this,
		bool, false, operator, throw,
		break, finally, out, true,
		byte, fixed, override, try,
		case, float, params, typeof,
		catch, for, private, uint,
		char, foreach, protected, ulong,
		checked, goto, public, unchecked,
		class, if, readonly, unsafe,
		const, implicit, ref, ushort,
		continue, in, return, using,
		decimal, int, sbyte, virtual,
		default, interface, sealed, volatile,
		delegate, internal, short, void,
		do, is, sizeof, while,
		double, lock, stackalloc,
		else, long, static,
		enum, namespace, string},
	keywordstyle=\color{cyan},
	identifierstyle=\color{red},
}
\usepackage{caption}
\DeclareCaptionFont{white}{\color{white}}
\DeclareCaptionFormat{listing}{\colorbox{blue}{\parbox{\textwidth}{\hspace{15pt}#1#2#3}}}
\captionsetup[lstlisting]{format=listing,labelfont=white,textfont=white, singlelinecheck=false, margin=0pt, font={bf,footnotesize}}



\addtolength{\hoffset}{-1.5cm}
\addtolength{\marginparwidth}{-1.5cm}
\addtolength{\textwidth}{3cm}
\addtolength{\voffset}{-1cm}
\addtolength{\textheight}{2.5cm}
\setlength{\topmargin}{0cm}
\setlength{\headheight}{0cm}

\begin{document}
	
        \title{Języki Skryptowe\\\small{Dokumentacja Projektu Szyfr Mendelejewa}}
        \author{Zuzanna Kłap, grupa 1/1\\Wydział Matematyki Stosowanej - Informatyka\\I semestr III}
        \date{\today}


	\maketitle
        \newpage
	\section*{Część I}
	\subsection*{Opis programu}
	\hspace{0.5cm} Algorytmion - edycja 2020 zadanie 3 
        
        Zadane było napisać program, który będzie szyfrował podany tekst w taki sposób, że zamiast kolejnych liter program będzie wypisywał liczbę atomową pierwiastka, którego symbolem chemicznym są szyfrowane litery bez uwzględnienia ich wielkości. Odstęp między literami zastępować miał symbol gwiazdki *, a odstęp między wyrazami - dwie gwiazdki **. 

        Niektóre teksty mogą być zaszyfrowane w ten sposób na różne sposoby,
        np. wyraz nos zaszyfrowany może być na 7*8*16 (azot, tlen, siarka), 102*16 (nobel, siarka) lub 7*76 (azot, osm). Program miał zwracać jedną z możliwości.

        Niektóre teksty nie mogą być zaszyfrowane w ten sposób, przykładem
        takiego wyrazu może być algorytmion. W takim wypadku program miał wypisać stosowny komunikat.


        
	\subsection*{Instrukcja obsługi}
        \hspace{0.5cm} Program jest aplikacją konsolową.

        Po jego uruchomieniu użytkownik jest proszony o wybór jednej z 3 opcji, poprzez wpisanie odpowiedniej cyfry i potwierdzenie klawiszem ENTER:

        1 - Uruchom projekt - Rozpoczyna się szyfrowanie danych które zostały wpisane do pliku input.txt i zostają one zapisane do pliku output.txt oraz jak backup. Następnie użytkownik proszony jest o wybranie, w jaki sposób chce on wyświetlić zaszyfrowane dane; jako wypis w konsoli, czy jako plik HTML.

        2 - Informacje i instrukcja obsługi - Użytkownik zostanie przekierowany do rozpisanej instrukcji i zasad dotyczących wpisywania danych. Program wypisuje też, w jaki sposób wyraz zostanie zaszyfrowany.

        3 - Zakończ - Opcja wyłącza program.
	


        \newpage
	\section*{Część II}
	\subsection*{Algorytmy}
	\hspace{0.5cm} Program pobiera dane z pliku wejściowego               input.txt oraz zapisuje je do pliku output.txt.
 
         Została zastosowana pętla while, przechodząca od indeksu 0      do takiego, który przyjmuje wartość długości wpisywanego przez użytkownika tekstu. Następnie użyte zostały instrukcje if, elif oraz else, które w połączeniu z pętlami for analizują wejściowy tekst oraz podejmują odpowiednie kroki z zaszyfrowaniem tekstu lub wyrzuceniem wyjątku w przypadku, gdy jest to niemożliwe.

        Dodatkowo utworzona została funkcja sprawdzająca, czy dany tekst można zaszyfrować, która wykorzystuje instrukcje try oraz except i wypisuje odpowiadające im rozwiązania.\\
        \\
        \begin{algorithm}[H]
            \KwData{Tekst wejściowy $text$}
            
            \SetKwFunction{CheckFunction}{check}
            \SetKwProg{Fn}{Function}{:}{}
            \Fn{\CheckFunction{$text$}}{
                \textbf{try} \{
                    $output \leftarrow$ \texttt{mendeleev\_cipher($text$)}\;
                    \texttt{Wydrukuj("Zaszyfrowany tekst:", $output$)}\;
                \} \textbf{except} Exception as e \{
                    \texttt{Wydrukuj("Nie można zaszyfrować podanego tekstu")}\;
                \}
            }
            \caption{Algorytm sprawdzający możliwość zaszyfrowania tekstu}
        \end{algorithm}
        
	\begin{algorithm}[H]
            \KwData{Dane wejściowe tekst $input\_text$}
            \KwResult{Tekst $output$ }
    
            $i=0$
        
            \While{$i < \text{len}(sentence)$}{
                \If{$\text{isalpha}(sentence[i])$}{
                    $Utworzenie pustej pary$\;
                    \If{$i + 1 < \text{len}(sentence)$}{
                        Przyspisanie parze wartości $sentence[i] + sentence[i + 1]$\;
                    }
        
                    \If{$pair$ jest w $elements$}{
                        Dodaj $elements[pair]$ do $ciphered$\;
        
                        \If{$i + 2 = \text{len}(sentence)$}{
                            \textbf{break}\;
                        } \Else{
                            Dodaj gwiazdkę do $ciphered$\;
                            Iteracja $i$\;
                        }
                    } \Else{
                        \If{$sentence[i]$ jest w $elements$}{
                            Dodaj $elements[sentence[i]]$ do $ciphered$\;
        
                            \If{$i + 1 = \text{len}(sentence)$}{
                                \textbf{break}\;
                            } \Else{
                                Dodaj gwiazdkę do $ciphered$\;
                            }
                        } \Else{
                            \textbf{raise Exception()}\;
                        }
                    }
                } \ElseIf{$sentence[i]$ to spacja}{
                    Dodaj gwiazdkę do $ciphered$\;
                } \Else{
                    Wydrukuj "Wprowadzono nieobsługiwane znaki"\;
                    \textbf{raise Exception()}\;
                }
        
                Iteracja $i$\;
            }
    
            \ForEach{$element$ \textbf{in} $ciphered$}{
                Dodanie $element$ do $output$\;
            }
        
            \Return $output$\;
            \caption{Algorytm szyfrowania Mendelejewa}
        \end{algorithm}

        \subsection*{Implementacja systemu}
	\hspace{0.5cm} Uruchomienie skryptu projekt.bat skutkuje             pojawieniem się menu głównego na ekranie.

        Po jego uruchomieniu użytkownik zostanie powitany, a następnie zostanie mu przedstawiony wybór akcji, które może podjąć;
        
        1 - Uruchom projekt,
        
        2 - Informacje i instrukcja obsługi,
        
        3 - Zakończ.
        
        Po wybraniu pierwszej opcji program zapyta się gdzie użytkownik chce wyświetlić wynik szyfrowania oraz wyświetli 2 opcje:

        1 - W konsoli,

        2 - W pliku HTML.

        Jeśli użytkownik wybierze opcję konsolową na ekranie zostanie wyświetlona informacja o wypisie danych z pliku output.txt, do którego to program zapisuje zaszyfrowane dane. Jeśli natomiast wybrana zostanie opcja druga, program przekieruje nas na stronę, w której do wglądu będziemy mieli pełny raport z szyfrowania.

        Użytkownik może też wybrać przeczytanie informacji i instrukcji obsługi. W takim przypadku zostanie wypisany komunikat "Podaj tekst, który chcesz zaszyfrować zgodnie z szyfrem Mendelejewa - zamiast kolejnych liter, wypisywane są liczby atomowe pierwiastka, odstęp między literami zastępuje symbol gwiazdki *, a odstęp między wyrazami dwie gwiazdki **. Zaszyfrowane nie mogą zostać znaki specjalne typu !,*,hashtag ani liczby, ponieważ mogłoby to wprowadzić błędy związane z odczytem zaszyfrowanego tekstu.".

        Po każdym z wyborów użytkownik ma możliwość powrotu do menu głównego albo zakończenia programu.

        \begin{center}
            \includegraphics[width=12cm]{implementacja systemu.png}
            
            Rysunek 1: Wizualizacja działania programu
        \end{center}

	\subsection*{Testy}
	\hspace{0.5cm} Aby przetestować poprawność programu należy go        uruchomić i wprowadzić do konsoli odpowiedni tekst.

        Na przykład, dla wpisanego tekstu "sobota rano" komputer wypisze 16*8*5*8*73**88*102, co po późniejszym porównaniu z tablicą Mendelejewa notujemy, jako poprawne.

        Aby sprawdzić, czy program odpowiednio wyłapuje błędy można wpisać "12Si!". Wtedy, zgodnie z prawdą program informuje nas, że "Wprowadzono nieobsługiwane znaki" oraz, że "Nie można zaszyfrować podanego tekstu", co pokrywa się z oczekiwaniami.

        Żeby sprawdzić, jak program radzi sobie z rozróżnianiem dużych i małych literek, przyjmijmy dla wpisywanego tekstu słowa "nos NOS". Program wypisze 102*16**102*16, co potwierdza poprawność jego działania.

        Ostatecznie, żeby sprawdzić, co się stanie, jeśli danego tekstu nie można zaszyfrować wpiszmy słowo "algorytmion". Tak, jak było przewidziane komputer wypisał "Nie można zaszyfrować podanego tekstu", a więc wszystko działa poprawnie.



	\subsection*{Pełen kod aplikacji}
        \centering projekt.py
\begin{lstlisting}
elements = {
    "H": 1, "HE": 2, "LI": 3, "BE": 4, "B": 5, "C": 6, "N": 7, "O": 8, "F": 9, "NE": 10,
    "NA": 11, "MG": 12, "AL": 13, "SI": 14, "P": 15, "S": 16, "CL": 17, "AR": 18, "K": 19,
    "CA": 20, "SC": 21, "TI": 22, "V": 23, "CR": 24, "MN": 25, "FE": 26, "CO": 27, "NI": 28,
    "CU": 29, "ZN": 30, "GA": 31, "GE": 32, "AS": 33, "SE": 34, "BR": 35, "KR": 36, "RB": 37,
    "SR": 38, "Y": 39, "ZR": 40, "NB": 41, "MO": 42, "TC": 43, "RU": 44, "RH": 45, "PD": 46,
    "AG": 47, "CD": 48, "IN": 49, "SN": 50, "SB": 51, "TE": 52, "I": 53, "XE": 54, "CS": 55,
    "BA": 56, "LA": 57, "CE": 58, "PR": 59, "ND": 60, "PM": 61, "SM": 62, "EU": 63, "GD": 64,
    "TB": 65, "DY": 66, "HO": 67, "ER": 68, "TM": 69, "YB": 70, "LU": 71, "HF": 72, "TA": 73,
    "W": 74, "RE": 75, "OS": 76, "IR": 77, "PT": 78, "AU": 79, "HG": 80, "TL": 81, "PB": 82,
    "BI": 83, "PO": 84, "AT": 85, "RN": 86, "FR": 87, "RA": 88, "AC": 89, "TH": 90, "PA": 91,
    "U": 92, "NP": 93, "PU": 94, "AM": 95, "CM": 96, "BK": 97, "CF": 98, "ES": 99, "FM": 100,
    "MD": 101, "NO": 102, "LR": 103, "RF": 104, "DB": 105, "SG": 106, "BH": 107, "HS": 108,
    "MT": 109, "DS": 110, "RG": 111, "CN": 112, "NH": 113, "FL": 114, "MC": 115, "LV": 116,
    "TS": 117, "OG": 118
}

def mendeleev_cipher(input_text):
    sentence = list(input_text.upper())
    ciphered = []
    output = ""
    
    i = 0
    while i < len(sentence):
        if sentence[i].isalpha():
            pair = "  "
            if i + 1 < len(sentence):
                pair = sentence[i] + sentence[i + 1]

            if pair in elements:
                ciphered.append(str(elements[pair]))

                if i + 2 == len(sentence):
                    break
                else:
                    ciphered.append("*")
                    i += 1
            else:
                if sentence[i] in elements:
                    ciphered.append(str(elements[sentence[i]]))
                    if i + 1 == len(sentence):
                        break
                    else:
                        ciphered.append("*")
                else:
                    raise Exception()
        elif sentence[i] == ' ':
            ciphered.append("*")
        else:
            print("Wprowadzono nieobslugiwane znaki")
            raise Exception()
        i += 1

    for element in ciphered:
        output += element

    return output

def check(text):
    try:
        output = mendeleev_cipher(text)
        print("Zaszyfrowany tekst:", output)
    except Exception as e:
        print("Nie mozna zaszyfrowac podanego tekstu")

if __name__ == "__main__":
    print("\nPodaj tekst, ktory chcesz zaszyfrowac zgodnie z szyfrem Mendelejewa - zamiast kolejnych liter, "
          "wypisywane sa liczby atomowe pierwiastka,\n"
          "odstep miedzy literami zastepuje symbol gwiazdki *, a odstep miedzy wyrazami dwie gwiazdki **.\n"
          "Zaszyfrowane nie moga zostac znaki specjalne typu !,*,# ani liczby, poniewaz mogloby to wprowadzic "
          "bledy zwiazane z odczytem zaszyfrowanego tekstu.\n")

    text = input("Tekst do zaszyfrowania: ")
    check(text)

\end{lstlisting}
\newpage

\hspace{0.5cm} projekt.bat
\begin{lstlisting}[style=batch, label=batchscript]
@echo off

echo Witaj w programie!

:menu
echo Wybierz opcje:
echo 1 - Uruchom projekt
echo 2 - Informacje i instrukcja obslugi
echo 3 - Zakoncz

set /p option=""

if "%option%"=="1" (
    call :run_project
) else if "%option%"=="2" (
    call :show_info
) else if "%option%"=="3" (
    echo Zakonczono program.
    exit /b
) else (
    echo Nieprawidlowy wybor. Sprobuj ponownie.
    goto menu
)

pause
exit /b

:run_project
python projekt.py
setlocal enabledelayedexpansion

:run_project_menu
echo Gdzie chcesz wyswietlic wynik szyfrowania?
echo 1 - W konsoli
echo 2 - W pliku HTML
echo 3 - Powrot do menu glownego

set /p choice=""

if "%choice%"=="1" (
    echo Wypisywanie danych z pliku output.txt:
    type output.txt
    goto :backing_up
) else if "%choice%"=="2" (
    echo Uruchamianie pliku HTML.
    start "" "http://127.0.0.1:5500/index.html"
    goto :backing_up
) else if "%choice%"=="3" (
    goto menu
) else (
    echo Wybor nieprawidlowy.
    goto run_project_menu
)

endlocal
goto :eof

:backing_up
echo 1 - Powrot do menu glownego
echo 2 - Zakoncz program

set /p choice=""
if "%choice%"=="1" (
    goto menu
) else if "%choice%"=="2" (
    echo Zakonczono program.
    exit /b
) else (
    echo Nieprawidlowy wybor.
    goto backing_up
)

:show_info
echo Instrukcja obslugi:
echo Podaj tekst, ktory chcesz zaszyfrowac zgodnie z szyfrem Mendelejewa - zamiast kolejnych liter,
echo wypisywane sa liczby atomowe pierwiastka,
echo odstep miedzy literami zastepuje symbol gwiazdki *, a odstep miedzy wyrazami dwie gwiazdki **.
echo Zaszyfrowane nie moga zostac znaki specjalne typu !,*,# ani liczby, poniewaz mogloby to wprowadzic
echo bledy zwiazane z odczytem zaszyfrowanego tekstu.
echo 1 - Powrot do menu glownego

set /p choice=""
if "%choice%"=="1" (
    goto menu
) else (
    echo Nieprawidlowy wybor.
    goto show_info
)

goto :eof

\end{lstlisting}
\noindent\hrulefill
\newpage
index.html
\begin{lstlisting}
    <!DOCTYPE html>
<html lang="en">
<head>
    <meta charset="UTF-8">
    <meta name="viewport" content="width=device-width, initial-scale=1.0">
    <title>Dynamic Table</title>
    <link rel="stylesheet" href="style.css">
    <script src="script.js" defer></script>
</head>
<body>
    <div id="datetime" class="datetime"></div>
    <h1>Szyfrowanie Mendelejewa </h1>
    <table id="dataTable">
        <thead>
            <tr>
                <th>Wiadomosc przed zaszyfrowaniem</th>
                <th>Wiadomosc zaszyfrowana</th>
            </tr>
        </thead>
        <tbody>
        </tbody>
    </table>
</body>
</html>
\end{lstlisting}
\newpage

script.js
\begin{lstlisting}[style=JavaScript, label=JabaScript]
document.addEventListener("DOMContentLoaded", function() {
    function updateTable() {
        fetch('input.txt')
            .then(response => response.text())
            .then(data1 => {
                return fetch('output.txt')
                    .then(response => response.text())
                    .then(data2 => {
                        updateTableBody(data1.split('\n'), data2.split('\n'));
                    });
            })
            .catch(error => console.error('Error fetching data:', error));

        var datetimeElement = document.getElementById('datetime');
        var currentDate = new Date();
        var formattedDate = currentDate.toLocaleDateString();
        var formattedTime = currentDate.toLocaleTimeString();
        datetimeElement.textContent = 'Raport z dnia ' + formattedDate + ' godzina ' + formattedTime;
    }

    function updateTableBody(values1, values2) {
        var tableBody = document.getElementById('dataTable').getElementsByTagName('tbody')[0];
        tableBody.innerHTML = '';

        for (var i = 0; i < values1.length; i++) {
            var row = tableBody.insertRow(i);
            var cell1 = row.insertCell(0);
            var cell2 = row.insertCell(1);

            cell1.textContent = values1[i];
            cell2.textContent = values2[i];
        }
    }

    updateTable();
});
\end{lstlisting}
\noindent\hrulefill
\newpage

style.cc
\begin{lstlisting}[style=CSS, label=CSS]
body {
    font-family: 'Arial', sans-serif;
    margin: 20px;
}

h1 {
    text-align: center;
}

table {
    width: 100%;
    border-collapse: collapse;
    margin-top: 20px;
}

th, td {
    border: 1px solid #ddd;
    padding: 8px;
    text-align: left;
}

th {
    background-color: #dce0e6;
}

tr:hover {
    background-color: #dce0e679;
}

.datetime {
    text-align: center;
}
\end{lstlisting}
\noindent\hrulefill

\end{document}
